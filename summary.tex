\documentclass[11pt]{article}%

\usepackage[top=1in,bottom=1in,left=1in,right=1in]{geometry}
\usepackage{url}
\usepackage{graphicx}
\usepackage[lofdepth,lotdepth]{subfig}
\usepackage{titlesec}
%\usepackage{timetabl}
%\usepackage{margins}
%\usepackage[dvips,usenames,dvipsnames]{color}

\pagestyle{empty}
% \titlespacing*{\section}
% {0pt}{2pt}{2pt}
% \titlespacing*{\subsection}
% {0pt}{0pt}{2pt}
% \titlespacing*{\subsubsection}
% {0pt}{0pt}{2pt}
\titlespacing*{\paragraph}
{0pt}{2pt}{2pt}

\begin{document}

\noindent{\bf Overview}

\noindent With this proposal, we would like to request 3 years of financial support for the development of QMCPy, a Quasi-Monte Carlo community software library in Python. Quasi-Monte Carlo (QMC) methods replace independent and identically distributed points by low discrepancy points, which can yield significant improvements in computational efficiency for a wide range of important problems, including machine learning, Bayesian inference, financial pricing, and uncertainty quantification. Despite the promise of such methods, there has yet to be a shared community effort in  existing software packages for QMC are primarily the initiatives of individual research groups. The primary objectives...

\noindent{\bf Intellectual Merit}

% \noindent The participants will present their research and will have an opportunity to learn about current research by others, in similar stages of their career. Such information will help young researchers broaden their views on current trends in statistics and probability. The participants will be strongly encouraged to attend all talks and posters at the different workshops, and will be requested to make their research accessible to a wide audience. The participants will also interact at the breaks, lunch and dinner, poster sessions and evening events, among themselves and also with senior researchers from academia and industry, who will play a mentoring role throughout the conference. The initial contact through a New Researchers conference will facilitate future interactions among participants at other scientific meetings (such as JSM), which can lead to new collaborations. The flagship New Researchers Conference is special, as it is organized by junior researchers for junior researchers, so as to better address their own needs. The Teaching workshop is co-organized by New Researchers and established educators to provide a broad set of perspectives on learning, teaching and pedagogy.

\noindent{\bf Broader Impacts}

% \noindent These conferences are explicitly aimed at training the next generation of leaders and researchers in probability, statistics and data science. In helping to create networks of new researchers, they lay the groundwork for future collaboration and the informal exchange of ideas and knowledge. They are also critical for building professional cohesion within the fields of statistics, probability and data science and setting new frontiers for research. Meeting people outside of one's research specialty area, and learning about their research, helps young researchers develop a more comprehensive research perspective, which is key for future editorial and administrative positions and professional services. The meetings also assist in general professional development in various critical areas of research, teaching, mentoring, publishing and funding. Under-represented groups (women, minorities, people with disabilities) are explicitly encouraged to attend in the conference announcements. By inspiring and guiding new researchers and also creating a sense of community amongst them, across research specialties, genders, ethnicities and geographical regions, these conferences significantly impacts the future of these fields. 

% section project_summary (end)
 
%\bibliographystyle{apalike}
%\bibliography{bibfile}{}
\end{document}