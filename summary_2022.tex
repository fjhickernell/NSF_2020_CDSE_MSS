\documentclass[11pt]{article}%

\usepackage[top=1in,bottom=1in,left=1in,right=1in]{geometry}
\usepackage{url}
\usepackage{graphicx}
\usepackage[lofdepth,lotdepth]{subfig}
\usepackage{titlesec}
\usepackage[dvipsnames]{xcolor}
%\usepackage{timetabl}
%\usepackage{margins}
%\usepackage[dvips,usenames,dvipsnames]{color}

\pagestyle{empty}
% \titlespacing*{\section}
% {0pt}{2pt}{2pt}
% \titlespacing*{\subsection}
% {0pt}{0pt}{2pt}
% \titlespacing*{\subsubsection}
% {0pt}{0pt}{2pt}
\titlespacing*{\paragraph}
{0pt}{2pt}{2pt}

\newcommand{\FJHNote}[1]{{\color{blue}Fred: #1}}
\newcommand{\SMNote}[1]{{\color{blue}Simon: #1}}
\newcommand{\SCTCNote}[1]{{\color{green}Sou-Cheng: #1}}
\renewcommand{\baselinestretch}{0.995}


\begin{document}

\noindent{\bf Overview} %Per NSF section headings need to be on a separate line

\noindent Simulations based on quasi-Monte Carlo (QMC) methods, also known as low discrepancy (LD) sampling, converge to the solution much faster than those based on independent and identically distributed (IID) sampling.  QMC has already had resounding success in several fields, including quantitative finance and partial differential equations with randomness. With rapid developments in science and engineering, there is the potential for QMC to make an even greater impact.
% solutions of (partial) differential equations with random coefficients
%My friends who do stochastic pdes say that they are not the same as pdes with randomness

This project will extend the scope of application for QMC sampling to important and timely areas, such as big data analytics, Bayesian computation, machine learning, and uncertainty quantification, through theoretical, methodological, and algorithmic advances. This will feature novel QMC developments in variation/variance reduction methods, automatic stopping criteria, big data subsampling methods, and low-discrepancy Bayesian sampling. We also will grow QMCPy, an open-source QMC Python software library that is gaining support among the QMC community. This will accelerate the adoption of QMCPy in new application areas and provide theoreticians and algorithm developers a test-bed of use cases for demonstrating their breakthroughs.

QMCPy will become a tightly connected collection of the best QMC software, including LD generators, QMC algorithms, and illuminating use cases. QMCPy will serve as a proving ground for new QMC methods, allowing researchers to test their ideas with the state-of-the-art on a broad range of use cases. QMCPy will be owned and cultivated by the QMC community and will provide an easy on-ramp for practitioners who are new to QMC.

% It will provide a user-friendly, consistent interface to the contributions of many scholars.

\bigskip

\noindent{\bf Intellectual Merit}

\noindent We will develop QMCPy and the theory that supports it in several novel and timely directions, guided by our motivation of bridging individual researchers to potential practitioners. We will improve the performance of QMCPy by adding flexibility and refactoring code for speed and to exploit GPUs and multi-core CPUs.  We will tackle the unresolved question of why the presumed theoretical convergence rate of $\mathcal{O}(n^{-3/2})$ is often not observed---even for smooth integrands---and explore variance reduction techniques to recover this rate in practice. We will investigate novel QMC sampling methods for multilevel approximation of expensive black-box functions. We will develop new LD methods for subsampling big data and demonstrate its effectiveness for scalable big data analytics. We will present a comprehensive framework for LD posterior sampling in complex Bayesian problems. Finally, to achieve our goal of bridging researchers to practitioners, we will show the benefits of these methods in solving cutting-edge problems for a range of scientific applications.

\bigskip

\noindent{\bf Broader Impacts}

\noindent By connecting individual QMC research groups with QMC practitioners, QMCPy will catalyze new application areas for low discrepancy sampling. QMCPy will serve as a proving ground for researchers to compare new ideas with the state-of-the-art on a broad range of realistic use cases. QMCPy will incorporate the best of other packages, such as SciPy, PyTorch, and TensorFlow, while featuring options that these other packages lack. QMCPy will offer practitioners a user-friendly platform for implementing the best QMC methods, which will bring about significant computational gains in many scientific disciplines. The successes and challenges of QMC in such areas will raise important theoretical questions and computational challenges, which this project will address.

QMCPy will serve as an important vehicle for educating cross-disciplinary researchers and promoting proper QMC practice. Students supported by this project will learn to write clean, efficient code that fits the package architecture, is documented, and passes doctests. They will also become proficient in software engineering tools, which are essential skills for computational mathematicians and statisticians. The products of this project -- software, academic articles, conference presentations, and documentation -- will also showcase the \textit{right} way to do low discrepancy sampling, both within the QMC community and to the broader scientific community.

% \noindent These conferences are explicitly aimed at training the next generation of leaders and researchers in probability, statistics and data science. In helping to create networks of new researchers, they lay the groundwork for future collaboration and the informal exchange of ideas and knowledge. They are also critical for building professional cohesion within the fields of statistics, probability and data science and setting new frontiers for research. Meeting people outside of one's research specialty area, and learning about their research, helps young researchers develop a more comprehensive research perspective, which is key for future editorial and administrative positions and professional services. The meetings also assist in general professional development in various critical areas of research, teaching, mentoring, publishing and funding. Under-represented groups (women, minorities, people with disabilities) are explicitly encouraged to attend in the conference announcements. By inspiring and guiding new researchers and also creating a sense of community amongst them, across research specialties, genders, ethnicities and geographical regions, these conferences significantly impacts the future of these fields. 

% section project_summary (end)
 
%\bibliographystyle{apalike}
%\bibliography{bibfile}{}
\end{document}