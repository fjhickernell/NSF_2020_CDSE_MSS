%Fred's and Sou-Cheng's grant 2018
\documentclass[11pt]{NSFamsart}
\usepackage{latexsym,amsfonts,amsmath,epsfig,multirow,stackrel,natbib,tabularx,hyperref,enumitem,xspace}
\usepackage[dvipsnames]{xcolor}
% This package prints the labels in the margin
%\usepackage[notref,notcite]{showkeys}


\pagestyle{empty}
\thispagestyle{empty}
%\thispagestyle{plain}
%\pagestyle{plain}

\headsep-0.40in
%\headsep-0.45in

\textwidth6.4in
\setlength{\oddsidemargin}{0in}
\setlength{\evensidemargin}{0in}
\textheight8.9in
%\textheight9.1in
\newcommand\myshade{85}
\colorlet{mylinkcolor}{violet}
\colorlet{mycitecolor}{Aquamarine}
%\colorlet{mycitecolor}{OliveGreen}
\colorlet{myurlcolor}{YellowOrange}

\hypersetup{
	linkcolor  = mylinkcolor!\myshade!black,
	citecolor  = mycitecolor!\myshade!black,
	urlcolor  = myurlcolor!\myshade!black,
	colorlinks = true,
}



%\setcounter{page}{1}


\begin{document}
%\setlength{\leftmargini}{2.5ex}

\centerline{\textbf{\Large Facilities, Equipment and Other Resources}}

\bigskip

\subsection*{Facilities}
All Illinois Tech faculty, PhD students, and visitors have offices provided at Illinois Tech.  Summer 
MS, BS, and high school students have shared work areas.  Faculty, student and visitor 
offices and conference rooms are provided by the Department of Applied Mathematics and the Office of Research.

The Department of Applied Mathematics has a research computer room that is available to all 
members of our department.  The Center for Interdisciplinary Scientific Computation (CISC)---of which PI Hickernell is a member---has a 256-core cluster named von Neumann funded by Illinois Tech.  Von Neumann is 
available available to all Illinois Tech research faculty and is
centrally managed by Illinois Tech Office of Technology (OTS) Services.  Illinois Tech is connected 
to the Open Science Grid through its own GridIIT.  

Illinois Tech has site licenses for Mathematica, MATLAB, SAS, and JMP.  Other open source 
software is also installed in our research and teaching laboratories.

Illinois Tech's university library provides access to journals, research monographs, and 
databases, either on-site, online or via inter-library loan.

\subsection*{Intellectual Resources}
\phantom{a}

\subsubsection*{Senior personnel}  
Dr.\ Sou-Cheng Choi, Research Asscoiate Professor at Illinois Tech and Principal Data Scientist at SAS, will provide in-kind, voluntary expertise in software engineering, documentation, and advising on important use cases.  She has co-authored numerous articles in the field of computational mathematics.  Dr.\ Choi is a co-author of the GAIL Matlab software library, a predecessor to QMCPy.

\subsubsection*{Unfunded collaborators} The PIs are part of a broad network of computational mathematicians and statisticians, including quasi-Monte Carlo (QMC) theorists and practitioners.  In the supplemental attachments, we have included letters of collaboration from a key collaborators on this project, should it be funded.

Prof.\ Art B. Owen (Stanford University) has engaged with the PIs in conversations about QMC for many years.  Prof.\ Owen is particularly expert in randomized QMC and the use of low discrepancy points for Markov chain Monte Carlo.  He has taken a keen interest in QMCPy and will put forward new QMC use cases, advise on software features to be included, and possibly collaborate on joint publications with the PIs. 

Dr.\ Michael J. McCourt (SigOpt) convinced his company to fund the early development of QMCPy.  He was convinced of the advantages of low discrepancy sampling and wanted to spread these ideas to the tech industry.  During the first two years of QMCPy's development, Mike advised us what we should prioritize for the benefit of tech practitioners.  Although SigOpt is not in a position to fund QMCPy further, Mike will advise us on the continued development of SigOpt.  He will also help us spread the word among his network in the machine learning community.

%Prof.\ Dirk Nuyens (Katholieke Universiteit Leuven) is an expert in fast construction of low discrepancy points.  He will work with us to build out QMCPy's capacity in that area and refactor our code for speed.

Prof.\ Chris Oates (Newcastle University) is an expert in probabilistic numerics and Stein discrepancies.  He will collaborate with us on our development of new stopping criteria and on ???

Dr.\ Jagadeeswaran Rathinavel (Wi-Tronix) has developed fast Bayesian stopping criteria for multivariate integration.  He will help us extend those stopping criteria to multifidelity problems.

Dr.\ Pieterjan Robbe (Sandia National Laboratories and Katholieke Universiteit Leuven) is a postdoctoral researcher who has contributed multilevel QMC capabilities and use cases in QMCPy.  He will extend those as well as contribute to the theory of multi-level and multifidelty model computation.

%Dr.\ Tim Sullivan (University of Warwick, UK) provides expertise in two application domains for the QMC software developed in this project.  The first area is probabilistic numerics, a Bayesian statistical approach to numerical tasks such as cubature and the solution of differential equations, in which the solution object is a statistical posterior distribution and reflects the discretization uncertainty inherited from the finite computational budget.  The second  is the use of QMC methods in the training of metamodels for heterogeneous (i.e. mixed atomistic-continuum) systems, which is an area of particular interest in Warwick's EPSRC Centre for Doctoral Training in Modelling of Heterogeneous Systems (HetSys).


\subsubsection*{Short-term visitors} Some scholars whose expertise would be immensely helpful to the goals of this project will be hosted at Illinois Tech and/or Duke University.

\bigskip \bigskip

Illinois Tech was listed on the National Federal Register of Historic Places in 2005. The proposed 
research activities will not make any physical changes to Illinois Tech's campus and buildings.


\end{document}
