%Fred's and Sou-Cheng's grant 2018
\documentclass[11pt]{NSFamsart}
\usepackage{latexsym,amsfonts,amsmath,epsfig,multirow,stackrel,natbib,tabularx,hyperref,enumitem,xspace}
\usepackage[dvipsnames]{xcolor}
% This package prints the labels in the margin
%\usepackage[notref,notcite]{showkeys}


%\pagestyle{empty}
\thispagestyle{plain}
\pagestyle{plain}

\headsep-0.6in
%\headsep-0.45in

\textwidth6.4in
\setlength{\oddsidemargin}{0in}
\setlength{\evensidemargin}{0in}
\textheight8.9in
%\textheight9.1in
\newcommand\myshade{85}
\colorlet{mylinkcolor}{violet}
\colorlet{mycitecolor}{Aquamarine}
%\colorlet{mycitecolor}{OliveGreen}
\colorlet{myurlcolor}{YellowOrange}

\hypersetup{
	linkcolor  = mylinkcolor!\myshade!black,
	citecolor  = mycitecolor!\myshade!black,
	urlcolor   = myurlcolor!\myshade!black,
	colorlinks = true,
}



%\setcounter{page}{1}


\begin{document}
%\setlength{\leftmargini}{2.5ex}

\centerline{\textbf{\Large Facilities, Equipment and Other Resources}}

\bigskip

\subsection*{Facilities}
All Illinois Tech faculty, PhD students, and visitors have offices provided at Illinois Tech.  Summer 
MS, BS, and high school students have shared work areas.  Faculty, student and visitor 
offices and conference rooms are provided by the Department of Applied Mathematics and the Office of Research.

The Department of Applied Mathematics has a research computer room that is available to all 
members of our department.  The Center for Interdisciplinary Scientific Computation (CISC)---of which PI Hickernell is a member---has a 256-core cluster named von Neumann funded by the 
College of Science.  Von Neumann is 
available available to all Illinois Tech research faculty and is
centrally managed by Illinois Tech Office of Technology (OTS) Services.  Illinois Tech is connected 
to the Open Science Grid through its own GridIIT.  

Illinois Tech has site licenses for Mathematica, MATLAB, SAS, and JMP.  Other open source 
software is also installed in our research and teaching laboratories.

Illinois Tech's university library provides access to journals, research monographs, and 
databases, either on-site, online or via inter-library loan.

\subsection*{Intellectual Resources}
\phantom{a}

\underline{Unfunded collaborators:} The PIs are part of a broad network of mathematical modelers and domain experts in statistics, numerical analysis and uncertainty quantification.  In the supplemental attachments, we have included letters of collaboration from our key collaborators on this project, and their roles in the current project, should the project be funded, are described.

Dr.\ Elizabeth Hunke (Los Alamos National Laboratory): Mac Hyman has collaborated with Dr.\ Hunke for more than twenty years climate modeling.  Dr.\ Hunke is the lead developer for the Los Alamos Sea Ice Model (CICE) and is responsible for development and incorporation of new parameterizations, model testing and validation, computational performance, documentation, and consultation with external model users on all aspects of sea ice modeling, including interfacing with global climate and earth system models.  She has agreed to collaborate with us in validating our approach in quantifying the uncertainty in the large-scale sea ice model used to predict global climate change. 


Dr.\  Kostas Tsigaridis (Goddard Institute for Space Studies)  Is an expert in large-scale models for understanding the role aerosols play in the Earth system.  His models 
capture the detailed aerosol processes such as the formation of secondary organic aerosols (SOA), to centennial time scale climate variability related to natural variability and the interactions and feedbacks between the atmosphere, the terrestrial biosphere, the ocean, and climate. Dr.\ Tsigaridis and Mac Hyman are co-mentoring a student, Helen Weierbach, in using these models to predict the impact that volcanoes will have on  North Atlantic Oscillation and global climate change.  He will be collaborating with us to use our adaptive sampling algorithm to sequentially identifying the input parameters for the simulations will provide the useful information.  

\underline{Shot-term visitors:} Some scholars whose expertise would be immensely helpful to the goals of this project will be hosted at Illinois Tech, Tulane U, and/or LANL.

\bigskip \bigskip

Illinois Tech was listed on the National Federal Register of Historic Places in 2005. The proposed 
research activities will not make any physical changes to Illinois Tech's campus and buildings.


\end{document}
